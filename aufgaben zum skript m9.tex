\documentclass[11pt,letterpaper, onecolumn, addpoints, answers]{exam}
\usepackage[ngerman]{babel}
\newcommand{\la}{\lambda}
\newcommand{\dl}{\Delta \lambda}
\newcommand{\Ra}{\Rightarrow}
\usepackage[utf8]{inputenc}
\usepackage[T1]{fontenc}
\usepackage{pgfplots}
\usepackage{xcolor}
%\usepackage{ntheorem}
\usepackage{chngcntr}
\usepackage{amsmath}
\usepackage{mathrsfs}
\usetikzlibrary{arrows}
\usepackage{mathtools}
\pgfplotsset{compat=1.15}
\usepackage{amssymb}
\usepackage{amsthm}
\usepackage{geometry}
\geometry{tmargin=25mm, bmargin=30mm, lmargin=20mm, rmargin=40mm}
\usepackage{mathtools}
\usepackage{drawmatrix}
%\theoremstyle{defintion}
\newtheorem {satz}{Satz}
\usepackage{pdfpages}
\bibliographystyle{abbrv}
\usepackage{lastpage}
\usepackage[fixFPpow]{tabularcalc}
\usepackage{xcolor}
\newtheorem{geg}{Gegeben}
\usepackage{graphicx}
\usepackage{hyperref}
\graphicspath{Bilder/}
\usepackage{chngcntr}
\usepackage{mathrsfs}

\usepackage{thmtools}

\usepackage{svg}

\checkboxchar{$\Box$}
\checkedchar{$\blacksquare$}

\renewcommand{\thequestion}{\thesection.\arabic{question}}
%\patchcmd{\questions}{10.}{\thequestion.}{}{}% fix left margin

%\renewcommand{\listtheoremname}{Definitionsverzeichnis}

\renewcommand{\d}{\text{d}}
\newcommand{\R}{\mathbb{R}}
\newcommand{\N}{\mathbb{N}}

\newcommand{\x}{\cdot}

\usepackage{multicol}
\usepackage{mdframed}
\usepackage{fancyhdr}

\renewenvironment{TheSolution}{\begin{mdframed}[skipabove=\baselineskip,innertopmargin=\baselineskip,innerbottommargin=\baselineskip]
\textbf{Lösung:}\enspace\ignorespaces}{\end{mdframed}}

\hsword{erreichte NP} 
\hpword{mögliche NP}
\htword{$\sum$}
\hqword{Aufgabe} 

\vsword{erreichte NP} 
\vpword{mögliche NP}
\vtword{$\sum$}
\vqword{Aufgabe} 

\renewcommand{\qedsymbol}{$\blacksquare$}

\renewcommand{\theequation}{\textsf{\arabic{section}.\arabic{equation}}}
\counterwithin*{equation}{section}
\counterwithin*{question}{section}
\usepackage{forest}
\usepackage{enumitem}
\usepackage{verbatim}

\setlength{\marginparwidth}{4cm}

\usepackage{titlesec}

\usepackage[sffamily]{roboto}

\usepackage{etoolbox}

\titleformat*{\section}{\centering \large\bfseries \sffamily}
\titleformat*{\subsection}{\large\bfseries}
\titleformat*{\subsubsection}{\large\bfseries}
\titleformat*{\paragraph}{\large\bfseries}
\titleformat*{\subparagraph}{\large\bfseries}

\pointformat{(\thepoints)}
\pointname{Be}

\runningheader{}{}{Name:\fillin[][1.5in]}
\runningfooter{}{}{\textsf{Seite \thepage\ von \numpages}}

\firstpageheader{}{}{Note (NP): \fillin[][0.75in]}
\firstpagefooter{}{}{\textsf{Seite \thepage\ von \numpages}}

\hpgword{Seite:}

\totalformat{\textsf{A\thequestion:\;\textbf{\totalpoints}}}

\title{Aufgaben zum Skript ``Mathematik 9''}
\author{Lukas Semrau}

%\renewcommand{\thesection}{\Roman{section}} 
\renewcommand{\thesubsection}{\thesection.\Roman{subsection}}

\printanswers
\renewcommand{\solutiontitle}{\noindent\textbf{Lösung:}\enspace}

\pointsinrightmargin
\pointsdroppedatright  %Self-explanatory
\boxedpoints

%\qformat{\thequestion:~(\thequestiontitle)}

  \definecolor{theme}{HTML}{383e79}
  \definecolor{term}{HTML}{009b8e}
  \definecolor{termref}{HTML}{008080}
  \definecolor{correct}{HTML}{383e79}

\qformat{\color{theme}\textbf{Aufgabe \thequestion: \thequestiontitle}\dotfill\totalpoints BE}

\usetikzlibrary{positioning}
\usetikzlibrary{calc}
\usetikzlibrary{backgrounds}
\usetikzlibrary{decorations.pathmorphing}
\usepgflibrary{arrows.meta}

\newtheoremstyle{def} % name
    {\topsep}                    % Space above
    {\topsep}                    % Space below
    {\small \sffamily}                   % Body font
    {}                           % Indent amount
    {\bfseries\sffamily \small }                   % Theorem head font
    {:}                          % Punctuation after theorem head
    {.5em}                       % Space after theorem head
    {}  % Theorem head spec (can be left empty, meaning ‘normal’)

\theoremstyle{def}
\newtheorem*{be}{zu den Punkten}

\begin{document}

{
    \begin{tikzpicture}[remember picture,overlay,anchor=north west,font=\normalsize]
      \coordinate (right bottom) at ([yshift=-2em]current page.north east);
      \coordinate (left top) at (current page.north west);
      \coordinate (right top) at (current page.north east);
      \path let \p1 = (right bottom), \p2 = (current page.north west) in
        coordinate (left bottom) at (\x2, \y1);

      \path (left top) -- (left bottom)
        node[midway, right, text=theme, align=left, xshift=1em]
          {Mathematik 9}
        -- (right bottom) -- (right top)
        node[midway, left, text=white, align=right, xshift=-1em](mw)
          {\textsf{\raisebox{-1pt}{lukas-semrau.de} }}
        node[below left=0 of mw](mwne) {};

      \path (left bottom) -- (right bottom)
        node[midway, below=4\baselineskip](titlepos) {};

      \path let \p1 = (mwne), \p2 = (right bottom) in coordinate (mw left bottom) at (\x1, \y2);

      \begin{scope}[on background layer]
        \fill[fill=theme, opacity=0.1] (left top) rectangle (mw left bottom);
        \fill[fill=theme] (mw left bottom) rectangle (right top);
      \end{scope}

      \node[anchor=north, font=\LARGE](title) at (titlepos)
        {Aufgaben zu  \textbf{\textcolor{theme}{Mathematik 9.}} \textit{Skript.}};
      \node[below=\baselineskip, font=\normalsize, align=center, inner sep=0.5em, text=black, text width=0.8\paperwidth](subtitle) at (title)
        {%
          Alle Aufgaben sind sauber, ordentlich und mit ausführlichem Rechenweg zu bearbeiten. \vspace{1em} \\*
          %
        };
      \node[below=2.8\baselineskip, font=\footnotesize, align=center](author) at (subtitle)
        {Lukas Semrau <lukas@lukas-semrau.de>};

      \node[below=4\baselineskip, align=center, text width=0.7\paperwidth, draw=theme, inner sep=2em, text=theme](example) at (author)
        {%
          \tableofcontents%
        };
        
        \node[below=20\baselineskip, align=center, text width=0.7\paperwidth, draw=theme, inner sep=2em, text=theme](example) at (author)
        {%
          Name \qquad \fillin[][5in]
        };

    \end{tikzpicture}
  }
\newpage

\section{Reelle Zahlen}
\begin{questions}

\titledquestion{Definitionsmengen angeben}[6] Geben sie jeweils die maximale Definitionsmenge $D$ der jeweiligen Terme an.
\begin{multicols}{3}
\begin{parts}
\part $\sqrt{x^2}$
\part $(\sqrt{x})^2$
\part $\sqrt{4-x^2}$
\part $\sqrt[3]{x}$
\part $\sqrt[2n]{x}$ mit $n\in \N$
\end{parts}
\end{multicols}
\begin{be}
Teilaufgabe (a) bis (d) werden jeweils mit 1Be bewertet, (e) wird aufgrund der Verallgemeinerung mit 2Be bewertet.
\end{be}
\titledquestion{Betragsstriche}
\begin{parts}
\part [3] Müssen Betragsstriche gesetzt werden? Kreuzen Sie an, wenn diese gesetzt werden müssen.
\begin{oneparcheckboxes}
\choice $\sqrt{x^2}$
\choice $\sqrt{x}^2$
\choice $\sqrt{x^3}$
\end{oneparcheckboxes}
\droppoints
\begin{be}
Jedes Kästchen wird mit 1Be bewertet, wird die richtige Antwort gegeben, erhält man diese Bewertungseinheit.
\end{be}
\part[2] Begründen Sie, weshalb folgender Satz nicht gilt: \textit{Für eine Zahl $a\in \R$ gilt: $\sqrt{a^2}=a$}. \droppoints
\part[8] Vereinfachen Sie. (Setzen Sie nur dann Betragsstriche, wenn sie unbedingt nötig sind!) \droppoints
\begin{subparts}
\begin{multicols}{4}
\subpart $\sqrt{16x^2}$
\subpart $\sqrt{3u^3}:\sqrt{u}$
\subpart $\sqrt{(-x-1)^2}$
\subpart $\sqrt{3a}\x \sqrt{3a}$
\end{multicols}
\end{subparts}
\begin{be}
Es können je 2Be erreicht werden.
\end{be}
\end{parts}
\titledquestion{Die Menge der reellen Zahlen / Zahlenmengen}
\begin{parts}
\part[6] Welche Rechenoperationen sind in der jeweiligen Zahlenmengen (zusätzlich zur vorherigen) möglich? \droppoints
\begin{subparts}
\subpart natürliche Zahlen:\quad \fillin[][3.5in]
\subpart ganze Zahlen:\quad \fillin[][3.5in]
\subpart rationale Zahlen:\quad \fillin[][3.5in]
\subpart reelle Zahlen:\quad \fillin[][3.5in]
\end{subparts}
\begin{be}
Jede richtige Rechenoperation ergibt 1Be.
\end{be}
\part[4] Welche Aussagen sind wahr? Kreuzen Sie jede richtige an.\droppoints
\begin{oneparcheckboxes}
\choice $0\in \N$
\choice $\sqrt{-3^2}\notin \R$
\choice $-1.6\overline{34}\in \mathbb Q$
\choice $\sqrt{121} \in \N$
\end{oneparcheckboxes}

\begin{be}
Jedes Kästchen wird mit 1Be bewertet, wird die richtige Antwort gegeben, erhält man diese Bewertungseinheit.
\end{be}
\end{parts}\pagebreak
\titledquestion{Umformungen und Vereinfachen}
\begin{parts}
\part[3] Welche Paare haben den gleichen Wert?\\ (Notieren Sie die Ergebnisse wie folgt: \textsf{ X - Y - Z}) \droppoints
\begin{oneparchoices}
\choice $\sqrt{112}$
\choice $\sqrt{7}\x \sqrt{28}$
\choice $9\sqrt{7}-\sqrt{63}$
\choice $42\sqrt{7}:3\sqrt{7}$
\choice $42/\sqrt{7}$
\choice $\sqrt{4\x2\x14}$
\choice $(\sqrt{7})^2\x (\sqrt{2})^2$
\end{oneparchoices}
\fillwithlines{1in}
\begin{be}
Jedes korrekte Paar wird mit einer Bewertungseinheit bewertet.
\end{be}
\part[9] Vereinfache. (machen Sie den Nenner ggf. rational; schreiben Sie immer als Wurzel, falls idese benötigt wird) \droppoints
\begin{subparts}
\subpart $(\sqrt{a-1}-\sqrt{a+1})(\sqrt{a+1}+\sqrt{a-1})$
\begin{multicols}{2}
\subpart $1/\sqrt{a}$
\subpart $(a-1):(\sqrt{a}+1)$
\subpart $\sqrt[n]{\sqrt[3]{a}}$
\subpart $\left( \sqrt[3]{2^4} \right)^\frac12$
\end{multicols}
\end{subparts}
\begin{be}
i, iii - v: 2Be // ii: 1Be
\end{be}
\end{parts}
\pagebreak
\section{Satzgruppe des Pythagoras}
\begin{figure}[h]
    \centering
    \caption{Das folgende rechtwinklige Dreieck wird im folgenden immer wieder betrachtet.}
    \definecolor{qqqqff}{rgb}{0,0,1}
\definecolor{qqwuqq}{rgb}{0,0.39215686274509803,0}
\definecolor{ccqqqq}{rgb}{0.8,0,0}
\begin{tikzpicture}[line cap=round,line join=round,>=triangle 45,x=1cm,y=1cm]
\clip(-0.5,-0.5) rectangle (5.5,2.5);
\draw[line width=2pt,color=ccqqqq,fill=ccqqqq,fill opacity=0.1] (3.7367336268050644,1.8683668134025322) -- (3.8683668134025324,1.6051004402075968) -- (4.131633186597468,1.7367336268050646) -- (4,2) -- cycle; 
\draw [shift={(5,0)},line width=2pt,color=qqwuqq,fill=qqwuqq,fill opacity=0.10000000149011612] (0,0) -- (116.56505117707799:0.5676282072462927) arc (116.56505117707799:180:0.5676282072462927) -- cycle;
\draw [shift={(0,0)},line width=2pt,color=qqqqff,fill=qqqqff,fill opacity=0.1] (0,0) -- (0:0.7568376096617235) arc (0:26.56505117707799:0.7568376096617235) -- cycle;
\draw [line width=2pt] (0,0)-- (4,2);
\draw [line width=2pt] (4,2)-- (5,0);
\draw [line width=2pt] (5,0)-- (0,0);
\draw (2.7112468590079843,-0.017280080088737635) node[anchor=north west] {$c$};
\draw (2.3,1.8) node[anchor=north west] {$b$};
\draw (4.565499002679207,1.3677327455922141) node[anchor=north west] {$a$};

\draw (3.6,1.6) node[anchor=north west] {$\gamma$};
\draw (4.1,0.65) node[anchor=north west] {$\beta$};
\draw (0.7,0.4) node[anchor=north west] {$\alpha$};
\begin{scriptsize}
\draw [fill=black] (0,0) circle (1.5pt) node [left] {\large $A$};
\draw [fill=black] (4,2) circle (1.5pt) node [above] {\large $C$};
\draw [fill=black] (5,0) circle (1.5pt) node [right] {\large $B$};
\end{scriptsize}
\end{tikzpicture}
    \label{fig:21}
\end{figure}

\titledquestion{Satz des Pythagoras}
\begin{parts}
\part[2] In einem Gleichschenkligen Dreieck mit 16 langer Basis misst die Höhe auf die Basis 15. Wie lange ist der Schenkel dieses Dreiecks. Fertige eine Skizze an. \droppoints
\part[3] Die basiswinkel eines gleichschnekligen Dreiecks betragen $60^\circ$, die Länge der Basis beträgt 5cm. Welche Länge hat die Höhe auf die Basis? \droppoints 
\part[2] Von einem Rechteck ist der Flächeninhalt $A=117.6$ und eine Seitenlänge $b=10.5$ gegeben. Berechnen Sie die Diagonale $d$ des Rechtecks. \droppoints
\part[3] In einem Prospekt ist Angegeben: \textit{``LCD-Fernseher 16:9, 80cm''} Dabei gibt 16:9 das Verhältnis der Seitenlängen an. Passt der Bildschirm in eine Schranknische der Länge 75cm? Begründen Sie. \droppoints
\part[6] Zeichnen die Punkte $P$ und $Q$ jeweil in ein Koordinatensystem. Und bestimmen Sie ihren Abstand durch Rechnung. \droppoints
\begin{subparts}
\subpart $P(2\mid 1);\quad Q(5\mid  5)$
\subpart finden Sie eine allgemeine Formel: $P(x_1\mid y_1);\quad Q(x_2\mid  y_2)$
\end{subparts}
\begin{be}
Für das korrekte KS: 1Be; für das Zeichnen der Punkte: 1Be; für die Rechnung in i: 1Be und in ii: 2Be.
\end{be}
\end{parts}
\titledquestion{Kehrsatz zum Satz des Pythagoras}
\begin{parts}
\part[3] Prüfen Sie, ob ein Dreieck mit den Seitenlängen $(a,b,c)$ Rechtwinklig ist. \droppoints
\begin{subparts}
\begin{multicols}{3}
\subpart $(28,45,53)$
\subpart $(16,62,64)$
\subpart $(286,290,48)$
\end{multicols}
\end{subparts}
\begin{be}
Je Teilaufgabe 1Be.
\end{be}
\part[4] Es gelten folgende Verhältnisse: $m^2-n^2$, $b=2mn$ und $c = m^2+n^2$. Beweisen Sie folgende Aussage: \textit{Für $m, n\in \N$ mit $m>n$ ist das Dreieck rechtwinklig.} \droppoints
\end{parts}\pagebreak
\titledquestion{Berechnungen von Seitenlängen}[9]
Berechne die fehlenden Größen.\footnote{Ein Beispieldreieck finden Sie unter \url{www.lukas-semrau.de/hohensatz-dreieck/}}
\begin{table}[h]
    \centering
    \begin{tabular}{c|c|c|c|c|c|c|c}
         &a&b&c&p&q&h&A  \\ \hline \hline
         a&&&&1.6&10.0&& \\ \hline 
         b&&&&4.0&&6.0&39.0 
    \end{tabular}
    \caption{Tabelle zur Aufgabe 2.3}
    \label{tab:my_label}
\end{table}
\titledquestion{Berechnungen an Figuren und Körpern.}
\begin{parts}
\part[2] Bestimmen Sie $d$. \droppoints
\begin{center}
\begin{tikzpicture}
\begin{axis}[
x=1cm,y=1cm,
axis lines=middle,
%ymajorgrids=true,
%xmajorgrids=true,
xmin=-1,
xmax=3.5,
ymin=-1,
ymax=3.5,
xtick={1,2,2.8284},
xticklabels={$1$,$2$, $d$},
ytick=\empty,
]

\draw[color = theme, line width = 1pt] (0,0) -- (2,0) -- (2,2) -- (0,2) -- cycle;
\draw[dashed, color = red, line width = 1pt] (0,0) -- (2,2);
\draw[dashed, color = red, line width = 1pt] (2.8284,0) arc (0:45:2.8284cm);
\end{axis}
\end{tikzpicture}
\end{center}
\part[6] Es wird eine Elfmeter auf ein Tor mit den Maßen $732\mathrm{cm}\times244\mathrm{cm}$.
Der Ball wird im oberen rechten Eck versenkt. Berechnen Sie den Weg, der den Ball zurückgelegt hat. \droppoints
\end{parts}
\pagebreak
\section{Quadratische Funktionen und Gleichungen}
\titledquestion{Parabeln verschieben}
\begin{parts}
\part[2] Bestimmen Sie den Funktionsterm einer Normalparabel, die \droppoints
\begin{subparts}
\subpart ... um 2 nach links und 3 nach oben verschoben ist.
\subpart ... um 3 nach unten und um 4 nach rechts verschoben ist.
\end{subparts}
\begin{be}
Je korrekten Term: 1Be
\end{be}
\part[3] Geben Sie den Funktionsterm an. \droppoints
\begin{subparts}
\begin{multicols}{3}
\subpart $S(-3\mid 0)$
\subpart $S(2.7\mid \sqrt{2})$
\subpart $S(-0.8\mid 2)$
\end{multicols}
\end{subparts}
\begin{be}
Je korrekten Term: 1Be
\end{be}
\part[4] Prüfen Sie, ob die Parabel $x\mapsto x^2-12+36$ eine in $x$-Richtung verschobene Normalparabel ist. \droppoints
\end{parts}
\titledquestion{Scheitel / Nullstellen}[12]
Bestimme den Scheitelpunkt und die Nullstellen des Graphens von $\mathrm{G}_f$.
\begin{parts}
\begin{multicols}{3}
\part $f_1(x)=(x+2)^2-9$
\part $f_2(x)=t^2+3t$
\part $f_3(x)=x^2+6x+4$
\end{multicols}
\end{parts}
\titledquestion{Textaufgaben}
\begin{parts}
\part[3]Das Produkt zweier positiven Zahlen, die sich um 6 unterscheiden, beträgt 6075. Wie lauten die beiden Zahlen? \droppoints
\part[4] Das Produkt zweier aufeinander folgender natürlicher Zahlen ist um 461 größer als ihre Summe. \droppoints
\part[3] Wurde in der Rechnnung wirklich ein Fehler gemacht? Begründen Sie. \droppoints
\begin{center}
\boxed{
\begin{aligned}
    2x^2-8x&= 0 \quad |:x \\
    2x-8&=0 \\
    x&=4 \\
    L&=\{4\} \quad \mathbf{f.}
\end{aligned}
}
\end{center}
\end{parts}
\titledquestion{Lineare Gleichungssystem mit drei Variablen}[9]
Lösen Sie das Gleichungssystem 
\begin{align}
    \begin{cases}
    x+4z&=4y \\
    2x+3z&=5y \\
    x&=y+z
    \end{cases}
\end{align}
\titledquestion{Funktionsterm bestimmen}[4]
Bestimmen Sie die Gleichung einer Parabel, die durch folgende Punkte geht:
$$P_1(-1\mid0), \quad P_2(2\mid 0)\quad \text{und} \quad P_3(3\mid 2)$$
\pagebreak
\titledquestion{Extremwertprobleme}
\begin{parts}
\part[2]Für welche ganze Zahl ist das Produkt aus Vorgänger und Nachfolger am kleinsten. \droppoints
\part[2]Für welche reelle Zahl ist das Produkt aus Vorgänger und Nachfolger am kleinsten. \droppoints 
\part[6] Gegeben ist eine Funktion 
\begin{align}
    f(x)=-\frac{1}{2}x+2.
\end{align}
Für $a\in ]0;4[$ mit $a\in \R$ werden folgende Punkte definiert: $O(0\mid 0)$, $P(a\mid f(a))$, $P_1(0\mid f(a))$ und $P_2(a\mid 0)$.\\ \medskip 
Wann ist der Flächeninhalt des Rechtecks $OP_2PP_1$ am größten? \droppoints
\begin{center}
\begin{tikzpicture}
\begin{axis}[
x=1cm,y=1cm,
axis lines=middle,
%ymajorgrids=true,
%xmajorgrids=true,
xmin=-1,
xmax=5,
ymin=-1,
ymax=3,
xtick={2},
xticklabels={\color{theme} $a$},
ytick={1},
yticklabels={\color{theme} $f(a)$}]
%Below the red parabola is defined
\addplot [
    domain=-1:5, 
    samples=100, 
    color=black,
]{-1/2*x+2};

\draw[color = theme,fill = theme!30, line width = 1pt] (0,0) -- (2,0) -- (2,1) -- (0,1) -- cycle;
\draw [fill = black, color = black] (2,1) circle (1pt) node[above right]{$(a\mid f(a)$};

\end{axis}
\end{tikzpicture}
\end{center}
\end{parts}
\titledquestion{Schnittprobleme}
\begin{parts}
\part[6]Geben Sie die Schnittpunkte der Funktionen \droppoints
\begin{align*}
    f(x)=\frac{1}{x}\quad \text{und} \quad g(x)=x-1.5\quad \text{an}.
\end{align*}
\part[9] Gegeben sind die Funktionen $f:x\mapsto -x^2+4x-3$ und $h_a:x\mapsto ax$. \droppoints
\begin{subparts}
\subpart Untersuchen Sie für welche  $a\in \R$ die Graphen einen, zwei oder keinen Schnittpunkt haben.
\subpart Bestimmen Sie die Koordinaten des Berührpunktes $B$ von $h_a$ und $f$.
\end{subparts}
\end{parts}
\pagebreak
\section{Wahrscheinlichkeit bei mehrstufigen Zufallsexperimenten}
%\qformat{\color{theme}\textbf{Aufgabe \thequestion:}\dotfill\totalpoints BE}
\titledquestion{Urnen} Eine Urne enthält zwei (identische) blaue Kugeln mit der Aufschrift ``T'' und zwei (identische) rote Kugeln mit der Aufschrift ``O''. Renate zieht aus der Urne zufällig die vier Kugeln. 
\begin{parts}
\part[4] Zeichne eine Baumdiagramm, das den Sachverhalt erklärt. \droppoints
\part[1]  Berechnen Sie die Wahrscheinlichkeit $p(\text{``Otoo''})$, dass das Wort ``Otto'' gezogen wird. \droppoints
\end{parts}
\titledquestion{Playlists}[4] In einer Playlist eines Smartphones befinden sich 20 Lieder, darunter genau fünf mit
deutschsprachigem Text. Die 20 Lieder werden in zufälliger Reihenfolge ohne Wiederholung
abgespielt. $p$ ist die Wahrscheinlichkeit dafür, dass die ersten drei gespielten Lieder einen
deutschsprachigen Text haben. \\
Erläutern Sie, warum der Ansatz $p=(1/4)^3$ falsch ist, und geben Sie einen richtigen Ansatz zur Berechnung von $p$ an.
\titledquestion{Basketballspiel}[4] Bei einem Basketballspieler sinkt die Trefferwahrscheinlichkeit bei einem Freiwurf während des 40-minütigen Spiels linear von anfänglich 70\% bis 50\%. Er bekommt nach der 8., 22., 38. Spielminute je einen Freiwurf zugesprochen. Mit welcher Wahrscheinlichkeit verwandelt er alle einen/zwei/drei Freiwürfe.
\titledquestion{Würfelwurf}[2] Ein normaler Würfel wird 6-mal geworfen. Mit welcher Wahrscheinlichkeit werden lauter verschiedene Augenzahlen geworfen.
\pagebreak
\section{Trigonometrie}
\titledquestion{Berechnungen an Dreiecken.}[9]
Berechnen Sie die fehlenden Seiten und Winkel eines rechtwinkligen Dreiecks ABC mit Hypotenuse $c$ und den Katheten $a$ und $b$. (Geben Sie dabei immer den vollständigen Rechenweg an.)
\begin{parts}
\begin{multicols}{3}
\part $b = 4.8;\; \beta = 42^\circ$
\part $c = 7.3;\; \alpha = 72^\circ$
\part $c = 3.3;\; \alpha = 5.7^\circ$
\end{multicols}
\end{parts}
\titledquestion{Beziehungen zwischen Sinus, Kosinus und Tangens.}
Vereinfachen Sie folgende Terme.
\begin{parts}
\begin{multicols}{3}
\part[2] $\sin \alpha - \sin \alpha \x \cos^2\alpha$
\part[3] $\frac{\sin^4\alpha-\cos^4\alpha}{\sin^2\alpha - \cos^2\alpha}$
\part[2] $1-\frac{1}{\sin^2\alpha}$
\end{multicols}
\end{parts}
\begin{be}
(a) 2; (b) 3; (c) 2
\end{be}
\titledquestion{Steigung}
Eine lineare Funktion besitzt den Funktionsterm $f:x\mapsto mx+t$. Die Steigung $m$ wird durch die Gleichung \begin{align}
    m&=\frac{\Delta y}{\Delta x} = \frac{f(b)-f(a)}{b-a}
\end{align}

\begin{center}
\begin{tikzpicture}
\begin{axis}[
x=1cm,y=1cm,
axis lines=middle,
ymajorgrids=true,
xmajorgrids=true,
xmin=-1,
xmax=5,
ymin=-1,
ymax=4,
xtick={2,4},
xticklabels={\color{theme} $a$, \color{theme} $b$},
ytick={2,3},
yticklabels={\color{theme} $f(a)$, \color{theme} $f(b)$}]
%Below the red parabola is defined
\addplot [
    domain=1.9:5, 
    samples=100, 
    color=black,
]{1/2*x+1};
\addplot [
    domain=-1:1.05, 
    samples=100, 
    color=black,
]{1/2*x+1};

\draw[color = theme,fill = theme!30, line width = 1pt] (2,2) -- (4,2) -- (4,3) -- cycle;
\draw [fill = black, color = black] (4,3) circle (1pt) node[above left]{$B(b\mid f(b)$};

\draw [fill = black, color = black] (2,2) circle (1pt) node[below, xshift = -0.1cm]{$A(a\mid f(a)$};

\draw [fill = black, color = black] (4,2) node[below right]{$C$};

\end{axis}
\end{tikzpicture}
\end{center}
\begin{parts}
\part[2]Zeigen Sie: $\tan \alpha = m$ \droppoints
\part[2]Berechne jeweils den Steigungswinkel der Geraden: \droppoints
\begin{multicols}{2}
\begin{subparts}
\subpart $y = 2x+5$
\subpart $y= \sqrt{3}x-2$
\end{subparts}
\end{multicols}
\begin{be}
Je 1Be
\end{be}
\part[6] Bestimme die Gleichung der Geraden, die durch den Punkt $P$ geht und den Steigungswinkel $\alpha$ hat. \droppoints
\begin{multicols}{2}
\begin{subparts}
\subpart $P(1\mid 2)$ und $\alpha = 30^\circ$
\subpart $P(-2\mid -3)$ und $\alpha = 60^\circ$
\end{subparts}
\end{multicols}
\begin{be}
Je 3Be
\end{be}
\end{parts}
\end{questions}
\end{document}
